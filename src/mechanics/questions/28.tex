\section{Теорема о сложении угловых скоростей твёрдого тела}

Рассмотрим $n+1$ репер $(O, \xvec{i}_p, \xvec{j}_p, \xvec{k}_p),~p \in [1:n+1]$
с центром в неподвижной точке $O$ твёрдого тела, и предположим, что первый и
последний реперы совпадают соответственно с неподвижным и подвижным реперами
$(O, \xvec{i}, \xvec{j}, \xvec{k})$ и $(O', \xvec{i}', \xvec{j}', \xvec{k}')$;
подвижный репер жёстко связан с движущимся твёрдым телом.

Пусть для всех $p \in [1:n]$ репер $(O, \xvec{i}_{p+1}, \xvec{j}_{p+1},
\xvec{k}_{p+1})$ движется относительно репера $(O, \xvec{i}_p, \xvec{j}_p,
\xvec{k}_p)$ с угловой скоростью $\xvec{\omega}_p$. В этом случае говорят, что
твёрдое тело совершает одновременное вращение с угловыми скоростями
$\xvec{\omega}_1, \dots, \xvec{\omega}_n$ вокруг осей
$\xvec{\omega}_1/\omega_1, \dots, \xvec{\omega}_n/\omega_n$.

Угловую скорость твёрдого тела (то есть угловую скорость подвижного репера
относительно неподвижного) обозначим $\xvec{\omega}$.

\begin{theorem}[Формула сложения угловых скоростей]
  Если твёрдое тело совершает одновременное вращение вокруг неподвижной точки с
  угловыми скоростями $\xvec{\omega}_1, \dots, \xvec{\omega}_n$, то его угловая
  скорость вычисляется по формуле
  \begin{equation}
    \xvec{\omega} = \xvec{\omega}_1 + \cdots + \xvec{\omega}_n.
  \end{equation}
\end{theorem}

\begin{proof}
  % TODO
  (\textcolor{red}{TODO:} доказать)
\end{proof}

\subsection{Список литературы}
\begin{enumerate}
  \item \cite{lectures}
\end{enumerate}

