\section{Движение механической системы. Твёрдое тело. Число степеней свободы
положения}

\subsection{Движение механической системы}

Пусть $T$ --- некоторое множество индексов $\tau$, которыми помечены все точки
механической системы, а $J \subset \mathbb{R}$ --- промежуток времени $t$, на
котором определено движение механической системы.

Пространством будем считать аффинное евклидово пространство $\mathbb{E}^n$;
точку этого пространства $M = (x,y,z) \in \mathbb{E}^n$ будем представлять
вектор-радиусом $\vec{r}$ в декартовой системе координат.

\begin{definition}
  \textit{Положением механической системы в момент времени $t_0$} будем называть
  семейство $\mathcal{M} = \{ M_\tau \}_{\tau \in T}$ точек в $\mathbb{E}^n$.
\end{definition}

\begin{definition}
  \textit{Движением механической системы} будем называть семейство
  $\mathcal{DM} = \{ D_\tau : J \to \mathbb{E}^n \}_{\tau \in T}$ дважды
  непрерывно дифференцируемых функций времени $t$ такое, что
  \begin{equation*}
    \forall \tau \in T \quad D_\tau(t_0) = M_\tau.
  \end{equation*}

  Ясно, что положением механической системы в любой другой момент времени
  $t \in J$ будет семейство $\{D_\tau(t)\}_{\tau \in T}$.
\end{definition}

\begin{definition}
  \textit{Перемещением механической системы} за время от $t_1$ до $t_2$ называют
  семейство векторов $\{ \vv{AB}~|~A = D_\tau(t_1),~ B = D_\tau(t_2) \}_{\tau 
  \in T}$.
\end{definition}

\subsection{Твёрдое тело}

\begin{definition}
  \textit{Классом движений} назовём некоторое множество движений $\mathcal{DM}$.
\end{definition}

\begin{definition}
  \textit{Неизменяемой на классе движений} назовём такую механическую систему,
  что
  \begin{equation*}
    \forall t \in J \quad \forall \tau_1, \tau_2 \in T \quad
      \rho(D_{\tau_1}(t), D_{\tau_2}(t)) = \rho(M_{\tau_1}, M_{\tau_2})
  \end{equation*}
  для любого движения этого класса.
\end{definition}

\begin{definition}
  Механическую систему назовём \textit{сплошной связной средой на классе
  движений}, если каждое её положение есть область или замкнутая область в
  $\mathbb{E}^n$.
\end{definition}

\begin{definition}
  \textit{Твёрдым телом} или \textit{абсолютно твёрдым телом на классе движений}
  назовём сплошную связную неизменяемую механическую систему на этом классе
  движений.
\end{definition}

\subsection{Число степеней свободы}

Будем говорить, что движение $\mathcal{DM} = \{ D_\tau \}_{\tau \in T}$ может
быть выражено через систему скалярных функций
\begin{equation*}
  q_i : J \to \mathbb{R}, \quad i = 1, \dots, m,
\end{equation*}
если
\begin{equation}
  \begin{array}{c c}
    \forall \tau \in T & \exists (q_1, \dots, q_m) \mapsto
      f_\tau(q_1, \dots, q_m) \\
    \forall t \in J & D_\tau(t) = f_\tau(q_1(t), \dots, q_m(t)).
  \end{array}
\end{equation}

\begin{definition}
  Говорят, что механическая система имеет \textit{$s$ степеней свободы положения
  на классе движений}, если всякое движение этого класса может быть выражено
  через некоторую систему скалярных функций
  \begin{equation*}
    q_i : J \to \mathbb{R}, \quad i = 1, \dots, s
  \end{equation*}
  и если хотя бы одно движение этого класса не может быть выражено ни через
  какую систему из меньшего числа скалярных функций.

  Если класс движений очевиден из контекста, то говорят просто о \textit{числе
  $s$ степеней свободы} механической системы.
\end{definition}

Рассмотрим механическую систему, состоящую из конечного числа $N$ точек. Такая
система на классе всех движений в $\mathbb{E}^n$ имеет $s = n \cdot N$ степеней
свободы.

Рассмотрим такой подкласс всех движений этой системы, для которых координаты
$(x_\nu, y_\nu, z_\nu),~\nu=1,\dots,N$ её точек удовлетворяют уравнениям
\begin{equation*}
  f_\nu(x_1, y_1, z_1, \dots, x_N, y_N, z_N) = 0, \quad \nu = 1,\dots,m,
\end{equation*}
причём фунции $f_\nu$ аргументов $(x_1,y_1,z_1, \dots, x_N,y_N,z_N)$ независимы
при $t \in J$; будем считать, что ранг матрицы Якоби этих функций равен $m$. В
этом случае говорят, что рассматривается механическая система из $N$ точек,
\textit{стеснённая $m$ голономными связями}.

\begin{theorem}
  Механическая система в $\mathbb{E}^n$ из $N$ точек, стеснённая $m$ голономными
  связями, имеет
  \begin{equation*}
    s = n \cdot N - m
  \end{equation*}
  степеней свободы.
\end{theorem}

\begin{proof}
  % TODO
  (\textcolor{red}{TODO:} доказать утверждение)
\end{proof}

\begin{theorem}
  Для твёрдого тела на классе всех его движений в $\mathbb{E}^n$ число степеней
  свободы положения равно
  \begin{equation*}
    s = \frac{n \cdot (n + 1)}{2}.
  \end{equation*}
\end{theorem}

\begin{proof}
  % TODO
  (\textcolor{red}{TODO:} доказать утверждение (указания можно найти на 37
  странице конспекта))
\end{proof}

\begin{definition}
  Движение твёрдого тела называют \textit{поступательным}, если у подвижного
  репера, связанного с этим телом, с течением времени может изменяться только
  начало.
\end{definition}

\begin{definition}
  Движение твёрдого тела называют \textit{вращением вокруг точки $O$}, если
  с течением времени не меняются координаты (в неподвижной системе) некоторой
  точки $O$ этого тела.
\end{definition}

% TODO
(\textcolor{red}{TODO:} найти число степеней свободы положения твёрдого тела на
этих двух классах движений)

\subsection{Список литературы}
\begin{enumerate}
  \item \cite{lectures}
\end{enumerate}

