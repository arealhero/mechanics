\section{Ускорение точек твёрдого тела в общем случае}

% TODO
(\textcolor{red}{TODO:} процесс дифференцирования)

Дифференцируя формулу Эйлера \ref{eq:general_euler_formula}, получим
\begin{equation}
  \label{eq:general_acceleration_temp}
  \vec{w} = \vec{w}_{O'}
    + \crossprod{\vec{\varepsilon}}{(\vec{r} - \vec{r}_{O'})}
    + \crossprod
      {\vec{\omega}}
      {(\crossprod{\vec{\omega}}{(\vec{r} - \vec{r}_{O'})})}.
\end{equation}

Первое слагаемое, $\vec{w}_{O'}$, определяет \textit{поступательное ускорение},
равное ускорению полюса, а второе
\begin{equation*}
  \vec{w}^\text{(в)} = \crossprod{\vec{\varepsilon}}{(\vec{r} - \vec{r}_{O'})}
\end{equation*}
и третье
\begin{equation*}
  \vec{w}^\text{(ос)} =
    \crossprod
      {\vec{\omega}}
      {(\crossprod{\vec{\omega}}{(\vec{r} - \vec{r}_{O'})})}
\end{equation*}
определяют \textit{вращательную} и \textit{осестремительную} составляющие
ускорения вращения тела вокруг полюса. Численные величины уже были исследованы и
выражаются формулами \ref{eq:immovable_point:rotational_acceleration_length} и
\ref{eq:immovable_point:centripetal_acceleration_length}.

\subsection{Список литературы}
\begin{enumerate}
  \item \cite{lectures}
\end{enumerate}

