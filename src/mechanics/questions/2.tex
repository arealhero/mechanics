\section{Аффинные координаты и преобразования}

\subsection{Аффинные и декартовы системы координат}

Пусть $E = \mathbb{E}^n$, тогда вектор $\vv{OM} \in \vec{E} = \mathbb{R}^n$
можно разложить по базису $(\vec{e}_1, \dots, \vec{e}_n)$ векторного
пространства $\mathbb{R}^n$:
\begin{equation}
  \label{eq:vec_coords}
  \vv{OM} = \sum_{j=1}^n x_j \vec{e}_j,
\end{equation}
или, в другой записи:
\begin{equation}
  \label{eq:vec_coords_alt}
  M = O + \sum_{j=1}^n x_j \vec{e}_j.
\end{equation}

Пусть $O \in \mathbb{E}^n$, а $(\vec{e}_1, \dots, \vec{e}_n)$ --- базис
пространства $\mathbb{R}^n$.
\begin{definition}
  Упорядоченную последовательность $(O, \vec{e}_1, \dots, \vec{e}_n)$ называют
  \textit{репером} пространства $\mathbb{E}^n$; точку $O$ называют
  \textit{началом} этого репера, а базис $(\vec{e}_1, \dots, \vec{e}_n)$ ---
  его \textit{базисом}.
\end{definition}

\begin{definition}
  Вещественные числа $x_1, \dots, x_n$ в \ref{eq:vec_coords_alt} называют
  \textit{аффинными координатами} точки $M \in \mathbb{E}^n$ относительно
  выбранного репера с началом $O \in \mathbb{E}^n$ и базисом
  $(\vec{e}_1, \dots, \vec{e}_n)$.
\end{definition}

\begin{definition}
  \textit{Ориентацией репера} называют ориентацию базиса соответствующего
  векторного пространства.
\end{definition}

% TODO: связь между репером, базисом и системой координат
\textcolor{red}{TODO:} связь между репером, базисом и системой координат.

\begin{definition}
  Аффинную систему координат, оси которой взаимно ортогональны, называют
  \textit{декартовой}.
\end{definition}

Пусть $(O, \vec{e}_1, \dots, \vec{e}_n)$ --- репер в пространстве
$\mathbb{E}^n$, и пусть даны представления точек $M,N \in \mathbb{E}^n$:

\begin{equation}
  \begin{aligned}
    M &= O + \sum_{j=1}^n x_j \vec{e}_j, \\
    N &= O + \sum_{j=1}^n y_j \vec{e}_j.
  \end{aligned}
\end{equation}

Тогда

\begin{equation}
  \begin{aligned}
    \vv{MN} &= \vv{MO} + \vv{ON} \\
    &= \vv{ON} - \vv{OM} \\
    &= \sum_{j=1}^n (y_j - x_j) \vec{e}_j.
  \end{aligned}
\end{equation}

\subsection{Аффинные преобразования}

Пусть
\begin{equation}
  M = O + \sum_{j=1}^n x_j \vec{e}_j = O_1 + \sum_{j=1}^n \tilde{x}_j \vec{e}_j,
\end{equation}
где
\begin{equation}
  O_1 = O + \sum_{j=1}^n a_j \vec{e}_j.
\end{equation}
Тогда
\begin{equation*}
  O + \sum_{j=1}^n x_j \vec{e}_j = O + \sum_{j=1}^n a_j \vec{e}_j +
    \sum_{j=1}^n \tilde{x}_j \vec{e}_j,
\end{equation*}
откуда следует, что
\begin{equation}
  x_j = \tilde{x}_j + a_j, \quad j \in [1:n].
\end{equation}

Рассмотрим два ортонормальных базиса $(\pvec{e}_1, \dots, \pvec{e}_n)$ и
$(\ppvec{e}_1, \dots,  \ppvec{e}_n)$ пространства $\mathbb{R}^n$. Как
известно, они связаны равенствами:
\begin{equation}
  \label{eq:basis_transform}
  \ppvec{e}_i = \sum_{j=1}^n p_{ij} \pvec{e}_j, \quad j \in [1:n].
\end{equation}

\begin{theorem}
  Матрица $P = (p_{ij})$ в \autoref{eq:basis_transform} ортогональна.
\end{theorem}

\begin{proof}
  Любое преобразование базисов вида \ref{eq:basis_transform} должно сохранять
  длины векторов, то есть
  \begin{equation*}
    \dotprod{\vec{x}}{\vec{x}} = \dotprod{P \vec{x}}{P \vec{x}} \quad \forall
      \vec{x} \in \mathbb{R}^n .
  \end{equation*}
  Так как
  \begin{equation*}
    \dotprod{P \vec{x}}{P \vec{x}} = \dotprod{\vec{x}}{P^T P \vec{x}},
  \end{equation*}
  а $P^T P$ --- симметричная матрица, то $P^T P = I$, что и является условием
  ортогональности.
\end{proof}

Из ортогональности матрицы $P$ следует, что
\begin{equation*}
  1 = \det I = \det (P^T P) = \det P^T \det P = (\det P)^2,
\end{equation*}
откуда $\det P = \pm 1$. Если элементы матрицы $P$ непрерывно зависят от
каких-то параметров, то и $\det P$ также непрерывно зависит от них. Отсюда
следует, что при изменении этих параметров величина $\det P$ не меняется.

Выразим теперь связь между координатами точки в различных реперах. Пусть
$\pvec{x} = (x_1', \dots, x_n')$ и $\ppvec{x} = (x_1'', \dots, x_n'')$ ---
разложения вектора $\vec{x}$ по базисам $(\pvec{e}_1, \dots, \pvec{e}_n)$ и
$(\ppvec{e}_1, \dots, \ppvec{e}_n)$ соответственно, тогда
\begin{equation*}
  \ppvec{x} = P \pvec{x}, \quad \pvec{x} = P^T \ppvec{x}.
\end{equation*}
Пусть теперь
\begin{equation*}
  M = O + \sum_{j=1}^n x_j' \pvec{e}_j = O_1 + \sum_{j=1}^n x_j'' \ppvec{e}_j,
\end{equation*}
где
\begin{equation*}
  O_1 = O + \sum_{j=1}^n a_j \pvec{e}_j.
\end{equation*}
Тогда из равенств
\begin{equation*}
  \begin{aligned}
    O + \sum_{j=1}^n x_j' \pvec{e}_j &= O + \sum_{j=1}^n a_j \pvec{e}_j +
      \sum_{i=1}^n x_i'' \ppvec{e}_i \\
    &= O + \sum_{j=1}^n a_j \pvec{e}_j +
      \sum_{j=1}^n \pvec{e}_j \sum_{i=1}^n p_{ij} x_i''
  \end{aligned}
\end{equation*}
следует, что
\begin{equation}
  x_j' = a_j + \sum_{i=1}^n p_{ij} x_i'', \quad j \in [1:n].
\end{equation}
Аналогично
\begin{equation}
  x_j'' = \sum_{i=1}^n p_{ji} (x_i' - a_i), \quad j \in [1:n].
\end{equation}

\subsection{Список литературы}
\begin{enumerate}
  \item \cite{lectures}
\end{enumerate}

