\section{Группа движений аффинного евклидова пространства}

\subsection{Предварительные сведения}

\begin{definition}
  \textit{Законом композиции на множестве $X$} называют отображение
  \begin{equation*}
    * : X \times X \to X.
  \end{equation*}
  Вместо $*(a,b)$ пишут $a * b$.
\end{definition}

\begin{definition}
  Пусть $*$ --- закон композиции на $X$. Тогда пару $(X, *)$ называют
  \textit{алгебраической структурой}.
\end{definition}

\begin{definition}
  Пусть $*$ --- закон композиции на $X$. Если
  \begin{equation*}
    \forall a,b,c \in X \quad a * (b * c) = (a * b) * c,
  \end{equation*}
  то закон композиции $*$ называется \textit{ассоциативным}.
\end{definition}

\begin{definition}
  Алгебраическая структура $(X, *)$ называется \textit{полугруппой}, если закон
  композиции $*$ ассоциативен.
\end{definition}

\begin{definition}
  Элемент $e \in X$ называется \textit{единичным} или \textit{нейтральным}
  относительно закона композиции $*$, если
  \begin{equation*}
    \forall x \in X \quad e * x = x * e = x.
  \end{equation*}
\end{definition}

\begin{remark}
  В алгебраической структуре $(X, *)$ не может быть более одного единичного
  элемента.
\end{remark}

\begin{definition}
  Полугруппу с единицей называют \textit{моноидом}.
\end{definition}

\begin{definition}
  Элемент $a$ моноида $(X, *, e)$ называют \textit{обратимым}, если
  \begin{equation*}
    \exists b \in X: \quad a * b = b * a = e.
  \end{equation*}
  Для элемента $b$ используют обозначение $a^{-1}$.
\end{definition}

\begin{definition}
  Моноид, все элементы которого обратимы, называют \textit{группой}.
\end{definition}

\begin{definition}
  Закон композиции $*$ называют \textit{коммутативным}, если
  \begin{equation*}
    \forall a,b \in X \quad a * b = b * a.
  \end{equation*}
\end{definition}

\begin{definition}
  Группу с коммутативным законом композиции называют \textit{абелевой}
  (\textit{коммутативной}) группой.
\end{definition}

\begin{definition}
  Подмножество $H$ группы $G$ называется \textit{подгруппой группы $G$}, если:
  \begin{enumerate}
    \item $H$ содержит единичный элемент из $G$:
      \begin{equation*}
        \quad e \in H;
      \end{equation*}
    \item $H$ содержит композицию любых двух элементов из $H$:
      \begin{equation*}
        \forall a,b \in H \quad a * b \in H;
      \end{equation*}
    \item $H$ содержит вместе со всяких своим элементом $h$ обратный к нему
      элемент $h^{-1}$:
      \begin{equation*}
        \forall h \in H \quad h^{-1} \in H.
      \end{equation*}
  \end{enumerate}
\end{definition}

Пусть $s(\Omega)$ --- множество всех биективных отображений $f : \Omega \to
\Omega$. Введём закон композиции $* : s(\Omega) \times s(\Omega) \to s(\Omega)$
такой, что
\begin{equation*}
  \forall \varphi, \psi \in s(\Omega) \quad \varphi * \psi = \varphi \circ \psi;
\end{equation*}
тогда $(s(\Omega), *)$ --- группа, причём её единицей является тождественное
отображение $\id_\Omega: \Omega \to \Omega$ такое, что
\begin{equation*}
  \forall x \in \Omega \quad \id_\Omega(x) = x.
\end{equation*}

\subsection{Группа движений твёрдого тела}

% TODO
(\textcolor{red}{TODO:} Дальше может быть путаница в терминах. Короче, надо
понять, что в его понимании такое "перемещение", но вот как я это понимаю.
Рассмотрим некоторое движение твёрдого тела $\mathcal{DM} = \{ D_\tau : J \to
\mathbb{E}^3 \}_{\tau \in T}$. Функция $D_\tau$ задаёт перемещение точки
$M_\tau$ твёрдого тела. У нас есть формула, по которому мы можем найти
коэффициенты $x_j^\tau(t)$, то есть перемещению точки соответствует биекция
$D : \mathbb{E}^3 \to \mathbb{E}^3$, задающаяся этой формулой. В этом случае
получается, что каждому движению твёрдого тела соответствует множество таких
биекций. Если это всё верно, то надо аккуратно переписать всё в верных
терминах.)

Рассмотрим движение $\mathcal{DM} = \{ D_\tau : J \to \mathbb{E}^3 \}_{\tau \in
T}$ механической системы в $\mathbb{E}^3$.

Пусть $(O, \vec{e}_1, \vec{e}_2, \vec{e}_3)$ --- некоторый фиксированный репер в
$\mathbb{E}^3$ и пусть
\begin{equation}
  \label{eq:coords_in_immovable_frame}
  M_\tau = D_\tau(t) = O + \sum_{j=1}^{3} x_j^\tau(t) \vec{e}_j, \quad
    \tau \in T.
\end{equation}

Так как свободное твёрдое тело (твёрдое тело на классе всех движений в
$\mathbb{E}^3$) имеет 6 степеней свободы, то функции $x_j^\tau(t)$ могут быть
выражены через какие-то 6 скалярных функций $q_1, \dots, q_6$.

етыре точки $M_0, M_1, M_2, M_3$ твёрдого тела выберем так, чтобы векторы
$\vv{M_0 M_1}, \vv{M_0 M_2}, \vv{M_0 M_3}$ образовывали ортонормированный базис
$(\vec{i}_1, \vec{i}_2, \vec{i}_3)$ пространства $\mathbb{R}^3$. Тогда каждая
точка $M_\tau$ твёрдого тела определяется своими аффинными координатами в репере
$(M_0, \vec{i}_1, \vec{i}_2, \vec{i}_3)$:
\begin{equation}
  \label{eq:coords_in_movable_frame}
  M_\tau = M_0 + \sum_{j=1}^{3} y_j^\tau \vec{i}_j,
\end{equation}
причём координаты $y_j^\tau$ не зависят от времени.

Векторы $\vec{i}_1, \vec{i}_2, \vec{i}_3$, построенные по движущимся точкам
$M_0, M_1, M_2, M_3$, являются функциями времени:
\begin{equation*}
  \vec{i}_j = \vec{i}_j(t), \quad j = 1,2,3.
\end{equation*}

Ортонормированные базисы $(\vec{e}_1, \vec{e}_2, \vec{e}_3)$ и $(\vec{i}_1(t),
\vec{i}_2(t), \vec{i}_3(t))$ пространства $\mathbb{R}^3$ связаны равенствами
\begin{equation}
  \vec{i}_k = \sum_{j=1}^{3} p_{kj}(t) \vec{e}_j, \quad k = 1,2,3,
\end{equation}
где матрица $P(t) = (p_{kj}(t))$ ортогональна.

Если $D_{M_0}$ --- движение точки $M_0$ и
\begin{equation*}
  D_{M_0}(t) = O + \sum_{j=1}^{3} a_j(t) \vec{e}_j,
\end{equation*}
то
\begin{equation}
  \label{eq:free_motion_point_coords}
  x_j^\tau(t) = a_j(t) + \sum_{k=1}^{3} p_{kj}(t) y_k^\tau, \quad j = 1,2,3.
\end{equation}

Элементы $p_{kj}$ ортогональной матрицы $P$ могут быть выражены через углы
Эйлера $\varphi, \psi, \theta$, поэтому формулы
\ref{eq:free_motion_point_coords} дают искомое представление для функций
$x_j^\tau$ через шесть функций $a_1(t), a_2(t), a_3(t), \varphi(t), \psi(t),
\theta(t)$. Это значит, что всякому перемещению соответствует биективное
отображение $D : \mathbb{E}^3 \to \mathbb{E}^3$, определяемое формулами
\ref{eq:free_motion_point_coords}.

Задавая всевозможные движения (то есть функции $a_1, a_2, a_3, \varphi, \psi,
\theta$) и фиксируя всевозможные моменты времени $t \in J$, мы будем получать те
или иные перемещения твёрдого тела за время от $t_0$ до $t$ и соответствующие
ему биекции $D : \mathbb{E}^3 \to \mathbb{E}^3$.

\begin{theorem}
  Семейство $D_3$ всех таких биекций является подгруппой группы
  $s(\mathbb{E}^3)$.
\end{theorem}

\begin{proof}
  % TODO
  (\textcolor{red}{TODO:} указания на странице 44 конспекта)
\end{proof}

\begin{definition}
  Семейство $D_3$ называют \textit{группой движений} в $\mathbb{E}^3$.
\end{definition}

\subsection{Подгруппы движений}

\begin{definition}
  Если матрица $P(t)$ не зависит от времени, то движение твёрдого тела называют
  \textit{поступательным}.
\end{definition}

Каждому перемещению твёрдого тела за время от $t_0$ до $t$ в некотором
поступательном движении соответствует некоторое множество биекций
$D : \mathbb{E}^3 \to \mathbb{E}^3$, определяемых формулами:
\begin{equation*}
  x_j^\tau(t) = a_j(t) + \sum_{k=1}^{3} p_{kj}^0 y_k^\tau, \quad j = 1,2,3,
\end{equation*}
где $P(t_0) = P^0 = (p_{kj}^0)$.

% Обозначим множество всевозможных таких биекций 

\begin{theorem}
  Множество $D_3^{\text{(п)}}$ всех таких биекций
\end{theorem}

\subsection{Список литературы}
\begin{enumerate}
  \item \cite{lectures}
\end{enumerate}

