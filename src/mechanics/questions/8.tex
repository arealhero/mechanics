\section{Определение кривизны траектории точки по движению}

\subsection{Кинематический метод}

Если известны модули скорости $v = v(t)$ и ускорения $w = w(t)$ движения точки,
то кривизну траектории можно найти по формулам:
\begin{equation}
  \begin{gathered}
    w_\tau = \dot{v}, \quad w_n = \sqrt{w^2 - w_\tau^2}, \\
    K = \frac{w_n}{v^2}, \quad \rho = \frac{1}{K}.
  \end{gathered}
\end{equation}

Если движение точки задано тройкой скалярных функций $x(t),~y(t),~z(t)$, то
\begin{equation}
  \begin{gathered}
    v = \sqrt{(\dot{x}(t))^2 + (\dot{y}(t))^2 + (\dot{z}(t))^2}, \\
    w = \sqrt{(\ddot{x}(t))^2 + (\ddot{y}(t))^2 + (\ddot{z}(t))^2}. \\
  \end{gathered}
\end{equation}

% TODO: можно ли здесь употреблять слово "ортогональных"?
Если же движение точки задано тройкой ортогональных криволинейных координат
--- скалярных функций $q_1(t),~q_2(t),~q_3(t)$, то проекции скорости и
ускорения точки выразятся по формулам \ref{eq:velocity_proj} и
\ref{eq:accel_proj} как
\begin{equation*}
  v_{q_m} = H_{q_m} \dot{q}_m, \quad w_{q_m} = \frac{1}{H_{q_m}} E_{q_m} (T),
    \quad m = 1,2,3.
\end{equation*}
Тогда
\begin{equation}
  \begin{gathered}
    v = \sqrt{(v_{q_1}(t))^2 + (v_{q_2}(t))^2 + (v_{q_3}(t))^2}, \\
    w = \sqrt{(w_{q_1}(t))^2 + (w_{q_2}(t))^2 + (w_{q_3}(t))^2}. \\
  \end{gathered}
\end{equation}

\subsection{Список литературы}
\begin{enumerate}
  \item \cite{lectures}
\end{enumerate}

