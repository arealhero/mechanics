\section{Натуральный триэдр. Проекции ускорения точки на оси натурального
триэдра}

\subsection{Натуральный триэдр траектории}

% TODO: картинки из книги
\textcolor{red}{TODO:} картинки из книги (страница 184)

Рассмотрим некоторую кривую, не лежащую в одной плоскости (кривую двоякой
кривизны). Установим на этой кривой начало $M_0$ и положительное направление
отсчёта дуг $\sigma$. Возьмём какую-нибудь текущую точку $M$, положение
которой определим либо дугой $\sigma$, либо вектор-радиусом $\vec{r}$
относительно некоторой неподвижной точки $O$. Через точку $M$ проведём
касательную к кривой; направление касательной в сторону возрастающих значений
$\sigma$ зададим единичным вектором касательной $\vec{\tau}$.

Возьмём на кривой весьма близкую к $M$ точку $M_1$; пусть положение её
определяется значением дуги $\sigma + \Delta \sigma$, причём
$\Delta \sigma > 0$, то есть $M_1$ лежит за $M$ в сторону положительного
отсчёта дуги. Единичный вектор касательной в точке $M_1$ обозначим через
$\vec{\tau}_1$. Проведём через $\vec{\tau}$ плоскость $\Pi$, параллельную
$\vec{\tau}_1$; чтобы построить её, достаточно перенести $\vec{\tau}_1$ в
точку $M$; два вектора $\vec{\tau}$ и $\vec{\tau}_1$, имеющие начало в точке
$M$, определяют положение $\Pi$. При изменении положения $M_1$ плоскость $\Pi$
также изменяет своё положение, вращаясь вокруг $\vec{\tau}$; если будем
приближать $M_1$ к $M$, уменьшая $\Delta \sigma$ до нуля, то эта плоскость
будет приближаться к некоторому предельному положению $\Pi_0$, называемому
\textit{соприкасающейся плоскостью}.

В точке $M$ проведём плоскость $N_0$, перпендикулярную к касательной. Эта
плоскость называется \textit{нормальной плоскостью} кривой. Любая прямая,
проведённая в этой плоскости через точку $M$, будет перпендикулярна к
$\vec{\tau}$, то есть будет \textit{нормальна} кривой; линия пересечения
нормальной и соприкасающейся плоскостей определяет \textit{главную нормаль}
кривой. Иными словами, главной нормалью называется нормаль, лежащая в
соприкасающейся плоскости. Нормаль, перпендикулярная к главной нормали,
называется \textit{бинормалью} кривой.

\begin{definition}
  Совокупность трёх взаимно перпендикулярных осей:
  \begin{enumerate}
    \item касательной, направленной в сторону возрастания дуги,
    \item главной нормали, направленной в сторону вогнутости кривой, и
    \item бинормали, направленной по отношению к касательной и главной нормали
      так же, как ось $Oz$ расположена по отношению к осям $Ox$ и $Oy$,
  \end{enumerate}
  образует так называемый \textit{натуральный триэдр}
  (естественный трёхгранник) кривой. Единичные векторы этих осей обозначим
  соответственно через $\vec{\tau}, \vec{n}$ и $\vec{b}$.
\end{definition}

Найдём выражения этих трёх единичных векторов натурального триэдра через
вектор-радиус точки на кривой, заданный как вектор-функция дуги:
\begin{equation}
  \vec{r} = \vec{r}(\sigma).
\end{equation}

Найдём прежде всего $\vec{\tau}$. По определению векторной производной вектор
$\der[\vec{r}]{\sigma}$ направлен по касательной к годографу вектора $\vec{r}$
в сторону возрастающих $\sigma$. С другой стороны, численная величина
производной равна
\begin{equation*}
  \abs{\der[\vec{r}]{\sigma}} = \frac{\abs{d \vec{r}}}{d \sigma} = 1.
\end{equation*}
Таким образом, векторная производная представляет собой искомый единичный
вектор касательной:
\begin{equation}
  \vec{\tau} = \der[\vec{r}]{\sigma}.
\end{equation}

Для определения единичного вектора главной нормали $\vec{n}$ обратимся к
рисунку. Рассмотрим равнобедренный треугольник, образованный векторами
$\vec{\tau}$ и $\vec{\tau}_1$ в плоскости $\Pi$. Если точка $M_1$ взята на
весьма малом расстоянии $\Delta \sigma$ от точки $M$, то угол $\varepsilon$
между касательными $\vec{\tau}$ и $\vec{\tau}_1$ в смежных точках кривой ---
его называют \textit{углом смежности} --- будет также мал и вектор
$\Delta \vec{r}$ с тем меньшей ошибкой, чем меньше $\Delta \sigma$, можно
считать перпендикулярным к $\vec{\tau}$ и, следовательно, параллельным вектору
нормали $\pvec{n}$, лежащему с $\Delta \vec{\tau}$ в одной и той же плоскости
$\Pi$. По величине $\abs{\Delta \vec{\tau}}$, как основание равнобедренного
треугольника с малым углом $\varepsilon$ при вершине и боковыми сторонами,
равными единице, будет равен
\begin{equation*}
  \abs{\Delta \vec{\tau}} = 2 \abs{\vec{\tau}} \sin \frac{\varepsilon}{2}
    \approx 2 \cdot 1 \cdot \frac{\varepsilon}{2} = \varepsilon.
\end{equation*}
Отсюда найдём (с точностью до малых высших порядков)
\begin{equation*}
  \Delta \vec{\tau} = \varepsilon \pvec{n},
\end{equation*}
или
\begin{equation*}
  \pvec{n} = \frac{1}{\varepsilon} \Delta \vec{\tau} =
    \deltader{\vec{\tau}}{\sigma} \cdot \frac{\Delta \sigma}{\varepsilon}.
\end{equation*}
Будем приближать $\Delta \sigma$ к нулю, тогда точка $M_1$ будет стремиться к
$M$, плоскость $\Pi$ --- к соприкасающейся плоскости $\Pi_0$, единичный вектор
нормали $\pvec{n}$ --- к искомому единичному вектору $\vec{n}$, и мы будем
иметь
\begin{equation*}
  \vec{n} = \lim_{\Delta \sigma \to 0} \deltader{\vec{\tau}}{\sigma} \cdot
    \lim_{\Delta \sigma \to 0} \frac{\Delta \sigma}{\varepsilon}.
\end{equation*}

Первый предел равен векторной производной
\begin{equation*}
  \der[\vec{\tau}]{\sigma} = \der{\sigma} \paren{\der[\vec{r}]{\sigma}} =
    \sder[\vec{r}]{\sigma};
\end{equation*}
что же касается второго предела, то заметим, что отношение
$\frac{\varepsilon}{\Delta \sigma}$, определяющее среднюю скорость поворота
касательной к кривой при переходе от данной точки к смежной, характеризует
\textit{среднюю кривизну} кривой на участке $(\sigma, \sigma + \Delta \sigma)$,
а величина
\begin{equation}
  \lim_{\Delta \sigma \to 0} \frac{\varepsilon}{\Delta \sigma} = K
\end{equation}
определяет \textit{кривизну} кривой в данной точке.

Таким образом, имеем следующее выражение единичного вектора \textit{главной
нормали}:
\begin{equation}
  \label{eq:main_norm}
  \vec{n} = \frac{1}{K} \der[\vec{\tau}]{\sigma} = \frac{1}{K}
    \sder[\vec{r}]{\sigma}.
\end{equation}
Величину $1/K = \rho$, имеющую размерность длины, называют \textit{радиусом
кривизны} кривой в данной точке.

В случае произвольной кривой через данную её точку и две смежные с нею точки
можно провести круг, который при стремлении смежных точек к данной
рассматриваемой будет стремиться к некоторому предельному кругу, называемому 
\textit{соприкасающимся кругом} или \textit{кругом кривизны}. Радиус этого круга
будет радиусом кривизны кривой, центр круга $C$ (\textcolor{red}{TODO:} ссылка
на картинку) --- \textit{центром кривизны} кривой. Очевидно, круг кривизны лежит
в соприкасающейся плоскости, центр кривизны $C$ --- на главной нормали со
стороны вогнутости кривой.

Введя радиус кривизны $\rho$, получим
\begin{equation}
  \vec{n} = \rho \der[\vec{\tau}]{\sigma} = \rho \sder[\vec{r}]{\sigma}.
\end{equation}

Теперь уже не составляет труда найти и единичный вектор бинормали. Из условия
выбора положительного направления на бинормали следует:
\begin{equation}
  \vec{b} = \crossprod{\vec{\tau}}{\vec{n}} = \frac{1}{K}
  \paren{\crossprod{\der[\vec{r}]{\sigma}}{\sder[\vec{r}]{\sigma}}} = \rho
  \paren{\crossprod{\der[\vec{r}]{\sigma}}{\sder[\vec{r}]{\sigma}}}.
\end{equation}

\subsection{Разложение ускорения по осям натурального триэдра траектории}

Обозначим через $v_\tau$ проекцию вектора скорости на направление касательной
к траектории. Очевидно, что $v_\tau$ по абсолютной величине равно численной
величине скорости $v$; что же касается знака $v_\tau$, то $v_\tau$ положительно,
если направление движения в данный момент совпадает с направлением
положительного отсчёта дуг $\sigma$ по траектории, и отрицательно в
противоположном случае. Будем иметь
\begin{equation}
  \label{eq:vel_natural}
  \vec{v} = v_\tau \vec{\tau}.
\end{equation}

Если $s$ --- пройденный путь, то $d \sigma = ds$, когда $d \sigma > 0$, и
$d \sigma = -ds$, если $d \sigma <0$, поэтому
\begin{equation}
  \label{eq:v_tau}
  v_\tau = \dt[\sigma] = \pm \dt[s] = \pm v.
\end{equation}

Вектор ускорения есть производная по времени от вектора скорости, поэтому
\begin{equation}
  \label{eq:acc_natural_temp}
  \vec{w} = \dt[\vec{v}] = \dt (v_\tau \vec{\tau}) = \dt[v_\tau] \vec{\tau} +
    v_\tau \dt[\vec{\tau}].
\end{equation}

Далее, имеем
\begin{equation*}
  \dt[\vec{\tau}] = \dt[\vec{\tau}] \dt[\sigma];
\end{equation*}
согласно формулам \ref{eq:main_norm} и \ref{eq:v_tau} найдём
\begin{equation*}
  \dt[\vec{\tau}] = \frac{1}{\rho} \vec{n} v_\tau.
\end{equation*}

Подставив полученное выражение в равенство \ref{eq:acc_natural_temp}, будем
иметь
\begin{equation}
  \label{eq:acc_natural}
  \vec{w} = \vec{\tau} \dt[v_\tau] + \vec{n} \frac{v^2}{\rho},
\end{equation}
где $v_\tau^2$ заменено на равное ему $v^2$.

Равенство \ref{eq:acc_natural} представляет собой \textit{разложение вектора
ускорения по осям натурального триэдра}.

Обозначим коэффициенты при единичных векторах $\vec{\tau},~\vec{n}$ и $\vec{b}$
в разложении \ref{eq:acc_natural}, то есть проекции ускорения на оси
натурального триэдра, соответственно через $w_\tau,~w_n$ и $w_b$; тогда будем
иметь
\begin{equation}
  \label{eq:acc_natural_general}
  \vec{w} = w_\tau \vec{\tau} + w_n \vec{n} + w_b \vec{b},
\end{equation}
причём из \autoref{eq:acc_natural} следует, что
\begin{equation*}
  w_\tau = \dt[v_\tau] = \ddt[\sigma], \quad w_n = \frac{v^2}{\rho},
    \quad w_b = 0.
\end{equation*}

Последнее равенство говорит о том, что вектор ускорения перпендикулярен к
бинормали, то есть \textit{ускорение лежит в соприкасающейся плоскости}.

Первое слагаемое в разложении \ref{eq:acc_natural_general}, $w_\tau \vec{\tau}$,
даёт \textit{касательную} (тангенциальную) составляющую ускорения, второе, $w_n
\vec{n}$, --- \textit{нормальную} составляющую ускорения. Иногда для краткости
их называют просто касательным и нормальным ускорениями.

Нормальное ускорение всегда совпадает по направлению с главной нормалью, так как
$w_n = \frac{v^2}{\rho}$ --- существенно положительная величина. Вспоминая ранее
сказанное о направлении $\vec{n}$, видим, что \textit{нормальное ускорение
направлено к центру кривизны траектории} (нормальное ускорение иногда ещё
называют поэтому \textit{центростремительным}), то есть по главной нормали к
траектории в сторону её вогнутости. Отсюда вытекает свойство ускорения:
\textit{вектор ускорения направлен в сторону вогнутости траектории}.

Итак, \textit{вектор ускорения в криволинейном движении может быть представлен
как геометрическая сумма двух ускорений: касательного и нормального}.

Величина ускорения может быть представлена так:
% TODO: dv/dt или dv_\tau/dt?
\begin{equation}
  w = \sqrt{w_\tau^2 + w_n^2} =
    \sqrt{\paren{\dt[v_\tau]}^2 + \frac{v^4}{\rho^2}},
\end{equation}
а направление задано косинусами углов, составляемых им с касательной и главной
нормалью к траектории:
\begin{equation}
  \cos(\widehat{\vec{w}, \vec{\tau}}) = \frac{w_\tau}{w}, \quad
  \cos(\widehat{\vec{w}, \vec{n}}) = \frac{w_n}{w}.
\end{equation}

\subsection{Список литературы}
\begin{enumerate}
  \item \cite{lourie}
\end{enumerate}

