\section{Локальные базисы криволинейных координат}

\subsection{Определение}

Обозначим через $\xvec{q} = (q_1, q_2, q_3)$ криволинейные координаты точки $M
\in D$, имеющей вектор-радиус $\xvec{r}$. Тогда можно записать
\begin{equation*}
  \xvec{r} = \xvec{r}(\xvec{q}) = \xvec{r}(q_1, q_2, q_3).
\end{equation*}

\begin{definition}
  Кривую, которую вычертит точка $M$, если изменять одну только координату
  $q_i$, а другим дать некоторые фиксированные значения, будем называть
  \textit{координатной линией $(q_i)$}.
\end{definition}

Через каждую точку $M_0$ с криволинейными координатами $\xvec{q}_0 = (q_{10},
q_{20}, q_{30})$ можно провести три координатные линии:
\begin{equation*}
  \begin{aligned}
    \mbox{линия~} (q_1):~\xvec{r} &= \xvec{r}(q_1,~q_{20},~q_{30}), \\
    \mbox{линия~} (q_2):~\xvec{r} &= \xvec{r}(q_{10},~q_2,~q_{30}), \\
    \mbox{линия~} (q_3):~\xvec{r} &= \xvec{r}(q_{10},~q_{20},~q_3).
  \end{aligned}
\end{equation*}

Если изменять сразу две координаты, а оставшуюся фиксировать, то получим
поверхности, заданные следующими уравнениями:
\begin{equation*}
  \begin{aligned}
    \mbox{поверхность~} (q_1 q_2):~\xvec{r} &= \xvec{r}(q_1,~q_2,~q_{30}), \\
    \mbox{поверхность~} (q_2 q_3):~\xvec{r} &= \xvec{r}(q_{10},~q_2,~q_3), \\
    \mbox{поверхность~} (q_3 q_1):~\xvec{r} &= \xvec{r}(q_1,~q_{20},~q_3).
  \end{aligned}
\end{equation*}

Эти поверхности будем называть \textit{координатными поверхностями}, а
касательные плоскости к ним в точке $M_0$ --- \textit{координатными
плоскостями}.

По определению, якобиан криволинейной системы координат отличен от нуля в
каждой точке области определения $G$. Векторы
$\diffp{\xvec{r}}{{q_1}},~\diffp{\xvec{r}}{{q_2}},~\diffp{\xvec{r}}{{q_3}}$
составляют строки матрицы этого якобиана и поэтому не могут быть нулевыми.

\begin{theorem}
  Векторы
  $\diffp{\xvec{r}}{{q_1}},~\diffp{\xvec{r}}{{q_2}},~\diffp{\xvec{r}}{{q_3}}$
  являются касательными соответственно к линиям
  $(q_1),~(q_2),~(q_3)$ в точке $\xvec{q}_0$.
\end{theorem}

\begin{proof}
  Для наглядности рассмотрим координатную линию $(q_1)$. Эта линия
  параметризуется переменной $q_1$ в точке $\xvec{q}_0$. Положим
  $\xvec{r} = \xvec{r}(q_1, q_{20}, q_{30})$, тогда производая
  $\diffp{\xvec{r}}{{q_1}}$ даст направление касательной к этой кривой в точке
  $\xvec{q}_0$.
\end{proof}

Касательные к координатным линиям в данной точке $M_0$, направленные в сторону
возрастания соответствующих координат, будем называть \textit{координатными
осями} $[q_1],~[q_2],~[q_3]$ в данной точке, а направление этих осей зададим
\textit{единичными векторами}
\begin{equation*}
    \xvec{k}_1 =
      \frac
        {\diffp{\xvec{r}}{{q_1}}}
        {\abs{\diffp{\xvec{r}}{{q_1}}}}, \quad
    \xvec{k}_2 =
      \frac
        {\diffp{\xvec{r}}{{q_2}}}
        {\abs{\diffp{\xvec{r}}{{q_2}}}}, \quad
    \xvec{k}_3 =
      \frac
        {\diffp{\xvec{r}}{{q_3}}}
        {\abs{\diffp{\xvec{r}}{{q_3}}}}.
\end{equation*}

\begin{definition}
  Совокупность векторов $(\xvec{k}_1, \xvec{k}_2, \xvec{k}_3)$ называют
  \textit{локальным базисом} криволинейной системы координат в точке $M_0$.
\end{definition}

\begin{definition}
  Если векторы $\xvec{k}_1, \xvec{k}_2, \xvec{k}_3$ взаимно ортогональны в точке
  $M_0$, то криволинейная система координат называется \textit{ортогональной} в
  этой точке.
\end{definition}

\subsection{Условие ортогональности}

Так как векторы $\diffp{\xvec{r}}{{q_1}},~\diffp{\xvec{r}}{{q_2}},
~\diffp{\xvec{r}}{{q_3}}$ ненулевые, то условия ортогональности локального базиса
\begin{equation*}
  \dotprod{\xvec{k}_i}{\xvec{k}_j} = 0, \quad i \neq j
\end{equation*}
эквивалентны равенствам
\begin{equation*}
  \dotprod{\diffp{\xvec{r}}{{q_i}}}{\diffp{\xvec{r}}{{q_j}}} = 0,
    \quad i \neq j
\end{equation*}
или, в координатной форме,
\begin{equation}
  \diffp{x}{{q_i}} \diffp{x}{{q_j}} + \diffp{y}{{q_i}} \diffp{y}{{q_j}} +
    \diffp{z}{{q_i}} \diffp{z}{{q_j}} = 0, \quad i \neq j.
\end{equation}

\subsection{Список литературы}
\begin{enumerate}
  \item \cite{lectures}
  \item \cite{lourie}
\end{enumerate}

