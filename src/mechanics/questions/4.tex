\section{Локальные базисы криволинейных координат}

Криволинейные координаты обозначим $\vec{q} = (q_1, q_2, q_3) \in
Q = \{ \vec{q} \, | \, \vec{q} = \vec{q}(\vec{r}), \vec{r} \in D \}$.

\subsection{Определение}

% TODO: инфа из учебника
\textcolor{red}{TODO:} инфа из учебника

\begin{definition}
  Пусть $\vec{q}_0 = (q_{10}, q_{20}, q_{30}) \in Q,~\vec{r}_0 =
  \vec{r}(\vec{q}_0) = (x_0, y_0, z_0)$, тогда множества
  \begin{equation}
    (q_{i0}) = \{ (x,y,z) \in D \, | \, q_i(x,y,z) = q_{i0} \}, \quad i = 1,2,3
  \end{equation}
  называют \textit{координатными поверхностями} криволинейной системы координат
  $\vec{q} = (q_1, q_2, q_3)$ в точке $(q_{10}, q_{20}, q_{30})$, а множества
  \begin{equation}
    \begin{aligned}
      \tilde{q}_1 &= (q_{20}) \cap (q_{30}) \\
      \tilde{q}_2 &= (q_{10}) \cap (q_{30}) \\
      \tilde{q}_3 &= (q_{10}) \cap (q_{20})
    \end{aligned}
  \end{equation}
  --- её \textit{координатными линиями} в этой точке.
\end{definition}

\begin{remark}
  $(q_{10}) \cap (q_{20}) \cap (q_{30}) = \{ (x_0, y_0, z_0) \}$.
\end{remark}

По определению, якобиан криволинейной системы координат отличен от нуля в
каждой точке области определения $Q$. Векторы
$\parder[\vec{r}]{q_1},~\parder[\vec{r}]{q_2},~\parder[\vec{r}]{q_3}$
составляют строки матрицы этого якобиана и поэтому не могут быть нулевыми.

\begin{theorem}
  Векторы $\parder[\vec{r}]{q_1},~\parder[\vec{r}]{q_2},~\parder[\vec{r}]{q_3}$
  являются касательными соответственно к линиям
  $\tilde{q}_1,~\tilde{q}_2,~\tilde{q}_3$ в точке $\vec{q}_0$.
\end{theorem}

\begin{proof}
  Для наглядности рассмотрим координатную кривую $\tilde{q}_1$. Эта кривая
  параметризуется переменной $q_i$ в точке $\vec{q}_0$. Положим
  $\vec{r} = \vec{r}(q_1, q_{20}, q_{30})$, тогда производая
  $\parder[\vec{r}]{q_1}$ даст направление касательной к этой кривой в точке
  $\vec{q}_0$.
\end{proof}

\begin{definition}
  Совокупность векторов $(\vec{\tau}_1, \vec{\tau}_2, \vec{\tau}_3)$,
  определяемых формулой
  \begin{equation*}
    \vec{\tau}_i = \frac{\parder[\vec{r}]{q_i}}{\abs{\parder[\vec{r}]{q_i}}},
      \quad i = 1,2,3
  \end{equation*}
  называют \textit{локальным базисом} криволинейной системы координат в точке
  $\vec{q}_0$.
\end{definition}

\begin{definition}
  Если векторы $\vec{\tau}_1,~\vec{\tau}_2,~\vec{\tau}_3$ взаимно ортогональны
  в точке $\vec{q}_0$, то криволинейная система координат называется
  \textit{ортогональной} в этой точке.
\end{definition}

\subsection{Условие ортогональности}

Так как векторы $\parder[\vec{r}]{q_1},~\parder[\vec{r}]{q_2},
~\parder[\vec{r}]{q_3}$ ненулевые, то условия ортогональности локального базиса
\begin{equation*}
  \dotprod{\vec{\tau}_i}{\vec{\tau}_j} = 0, \quad i,j = 1,2,3,~i \neq j
\end{equation*}
эквивалентны равенствам
\begin{equation*}
  \dotprod{\parder[\vec{r}]{q_i}}{\parder[\vec{r}]{q_j}} = 0,
    \quad i,j = 1,2,3,~i \neq j
\end{equation*}
или, в координатной форме,
\begin{equation}
  \parder[x]{q_i} \parder[x]{q_j} + \parder[y]{q_i} \parder[y]{q_j} +
    \parder[z]{q_i} \parder[z]{q_j} = 0, \quad i,j = 1,2,3,~i \neq j.
\end{equation}

\subsection{Список литературы}
\begin{enumerate}
  \item \cite{lectures}
\end{enumerate}

