\section{Теорема сложения скоростей в сложном движении точки}

\begin{theorem}[Формула сложения скоростей]
  Абсолютная, переносная и относительная скорости движения точки связаны
  равенством
  \begin{equation}
    \xvec{v} = \xvec{v}_e + \xvec{v}_r
  \end{equation}
\end{theorem}

\begin{proof}
  Так как $\xvec{r} = \xvec{r}_0 + \xvec{r}'$, то, применяя формулу
  относительной производной \ref{eq:relative_derivative}, получаем
  \begin{equation*}
    \xvec{v} = \dt[\xvec{r}] = \xvec{v}_0
      + \crossprod{\xvec{\omega}}{\xvec{r}'}
      + \frac{d' \xvec{r}}{dt}.
  \end{equation*}

  По формуле Эйлера скорость той точки $M'$ подвижного пространства, с которой в
  данный момент $t$ совпадает движущаяся точка $M$, равна
  \begin{equation*}
    \xvec{v}_e = \xvec{v}_0 + \crossprod{\xvec{\omega}}{\xvec{r}'}.
  \end{equation*}
  Учитывая, что $\frac{d' \xvec{r}}{dt} = \xvec{v}_r$, получаем требуемое
  равенство.
\end{proof}

\subsection{Список литературы}
\begin{enumerate}
  \item \cite{lectures}
\end{enumerate}

