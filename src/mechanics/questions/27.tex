\section{Теорема сложения ускорений в сложном движении точки}

\begin{definition}
  Вектор
  \begin{equation*}
    \xvec{w}_c = 2 \crossprod{\xvec{\omega}}{\xvec{v}_r}
  \end{equation*}
  называют \textit{ускорением Кориолиса} (\textit{вращательным ускорением})
  точки в её сложном движении.
\end{definition}

\begin{theorem}[Формула Кориолиса сложения ускорений]
  \label{theorem:coriolis_formula}
  Абсолютное, переносное, относительное и вращательное ускорения в сложном
  движении точки связаны равенством
  \begin{equation}
    \label{eq:acceleration_summation}
    \xvec{w} = \xvec{w}_e + \xvec{w}_r + \xvec{w}_c.
  \end{equation}
\end{theorem}

\begin{proof}
  Дифференцируя формулу сложения скоростей \ref{eq:velocity_summation}, получаем
  \begin{equation}
    \label{eq:velocity_summation_derivative}
    \xvec{w} = \dt[\xvec{v}] = \dt[\xvec{v}_e] + \dt[\xvec{v}_r].
  \end{equation}

  Из формулы относительной производной \ref{eq:relative_derivative} следует, что
  \begin{equation*}
    \begin{aligned}
      \dt[\xvec{v}_r] &= \frac{d' \xvec{v}_r}{dt} +
        \crossprod{\xvec{\omega}}{\xvec{v}_r} \\
      &= \xvec{w}_r + \crossprod{\xvec{\omega}}{\xvec{v}_r}.
    \end{aligned}
  \end{equation*}

  Пусть $\xvec{\varepsilon} = \dt[\xvec{w}]$ --- угловое ускорение подвижного
  репера. По формуле Эйлера
  \begin{equation*}
    \xvec{v}_e = \xvec{v}_0
      + \crossprod
        {\xvec{\omega}}
        {(\xvec{r} - \xvec{r}_0)},
  \end{equation*}
  поэтому, используя формулу $\xvec{v} = \xvec{v}_e + \xvec{v}_r$, приходим к
  равенствам
  \begin{equation*}
    \begin{aligned}
      \dt[\xvec{v}_e] &= \xvec{w}_0
        + \crossprod{\xvec{\varepsilon}}{(\xvec{r} - \xvec{r}_0)}
        + \crossprod{\xvec{\omega}}{(\xvec{v} - \xvec{v}_0)} \\
      &= \xvec{w}_0
        + \crossprod{\xvec{\varepsilon}}{(\xvec{r} - \xvec{r}_0)}
        + \crossprod{\xvec{\omega}}{(\xvec{v}_e - \xvec{v}_0)}
        + \crossprod{\xvec{\omega}}{\xvec{v}_r}.
    \end{aligned}
  \end{equation*}

  Из формулы \ref{eq:general_acceleration_temp} следует, что
  \begin{equation*}
    \dt[\xvec{v}_e] = \xvec{w}_e + \crossprod{\xvec{\omega}}{\xvec{v}_r}.
  \end{equation*}

  Подставляя полученные равенства в формулу
  \ref{eq:velocity_summation_derivative}, окончательно получаем
  \begin{equation*}
    \begin{aligned}
      \xvec{w} &= \dt[\xvec{v}_e] + \dt[\xvec{v}_r] \\
      &= \xvec{w}_e + \crossprod{\xvec{\omega}}{\xvec{v}_r}
        + \xvec{w}_r + \crossprod{\xvec{\omega}}{\xvec{v}_r} \\
      &= \xvec{w}_e + \xvec{w}_r + 2 \crossprod{\xvec{\omega}}{\xvec{v}_r} \\
      &= \xvec{w}_e + \xvec{w}_r + \xvec{w}_c.
    \end{aligned}
  \end{equation*}
\end{proof}

\subsection{Список литературы}
\begin{enumerate}
  \item \cite{lectures}
\end{enumerate}

