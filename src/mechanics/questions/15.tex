\section{Две геометрические теоремы о плоском движении}

\begin{theorem}
  Всякое перемещение плоской фигуры в своей плоскости, а следовательно, и всякое
  плоское перемещение твёрдого тела можно себе представить как совокупность двух
  перемещений:
  \begin{enumerate}
    \item поступательного перемещения, зависящего от выбора полюса, и
    \item вращательного перемещения вокруг полюса;
  \end{enumerate}
  угол и направление поворота от выбора полюса не зависят.
\end{theorem}

\begin{proof}
  Положение плоской фигуры может быть задано положением двух её точек $O'$ и
  $M$ или положением отрезка $O'M$
  % TODO
  (\textcolor{red}{TODO:} рисунок 149, стр. 234)

  Пусть фигура $O'M$ переместилась из положения $I$ в положение $II$. Разобьём
  переход на две части. Сначала переместим фигуру поступательно в положение
  $I'$, причём все точки её получат перемещения, геометрические равные
  перемещению $\vv{O'O_1}$ полюса $O'$, а затем повернём фигуру на $\angle M'O_1
  M_1$ вокруг оси, проходящей через точку $O_1$ перпендикулярно к плоскости
  фигуры.

  Заметим, что вектор поступательного перемещения зависит от выбора полюса, а
  угол поворота не зависит от этого выбора. В самом деле, тот же переход из
  положения $I$ в положение $II$ можно осуществить, приняв за полюс точку $M$ и
  переместив сначала фигуру в положение $II'$ (\textcolor{red}{TODO:} картинка),
  причём все точки фигуры получат перемещения, геометрически равные
  $\vv{M M_1}$ и отличные от $\vv{O' O_1}$, а затем повернув фигуру на
  $\angle O'' M_1 O_1$ вокруг оси, проходящей через $M_1$. Но по свойству
  поступательного перемещения $\vv{O'' M_1}$ параллелен $\vv{O' M}$ и точно так
  же 
\end{proof}

\subsection{Список литературы}
\begin{enumerate}
  \item \cite{lourie}
\end{enumerate}

