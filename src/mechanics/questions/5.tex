\section{Коэффициенты Ламе. Проекции скорости точки на оси криволинейной
системы координат}

\subsection{Общие сведения}

В качестве пространства будем использовать аффинное евклидово пространство
$\mathbb{E}^n$.

\begin{definition}
  \textit{Положением механической системы в момент $t_0$} будем называть точку
  $M^0 \in \mathbb{E}^n$.
\end{definition}

\begin{definition}
  Пусть $J$ --- промежуток на $\mathbb{R}$. \textit{Движением} механической
  системы будем называть дважды непрерывно дифференцируемую функцию $D : J \to
  \mathbb{E}^n$ времени $t$ такую, что $D(t_0) = M^0$.
\end{definition}

\begin{definition}
  Предположим, что точка этого пространства может быть задана радиус-вектором
  $\xvec{r}$ в какой-либо декартовой системе координат, то есть движение этой
  точки представлено вектор-функцией $\xvec{r} : J \to \mathbb{R}^n$. В этом
  случае \textit{скоростью} и \textit{ускорением} точки в этом движении
  называют соответственно вектор-функции $\xvec{v} = \xvec[.]{r}$ и
  $\xvec{v} = \xvec[:]{r}$, а \textit{траекторией} точки называют кривую
  $\{ \xvec{r}(t) \in \mathbb{R}^n \, | \, t \in J \}$.
\end{definition}

% TODO | FIXME: кривая представлена в виде { r \in R | t \in J }. Необходимо
%   перенести условие из левой части задания множества в правую,
%   т.е. избавиться от r \in R слева.

\subsection{Коэффициенты Ламе}

Так как
\begin{equation}
  \diffp{\xvec{r}}{{q_m}} = \diffp{x}{{q_m}} \xvec{i} + \diffp{y}{{q_m}} \xvec{j} +
  \diffp{z}{{q_m}} \xvec{k},
\end{equation}
то, введя обозначение
\begin{equation}
  H_m = \abs{\diffp{\xvec{r}}{{q_m}}} = \sqrt{\paren{\diffp{x}{{q_m}}}^2 +
  \paren{\diffp{y}{{q_m}}}^2 + \paren{\diffp{z}{{q_m}}}^2},
\end{equation}
векторы локального базиса можно представить в виде
\begin{equation}
  \label{eq:local_basis_vec}
  \xvec{k}_m = \frac{1}{H_m} \diffp{\xvec{r}}{{q_m}},
\end{equation}
или, иначе:
\begin{equation}
  \label{eq:local_basis_vec_alt}
  \diffp{\xvec{r}}{{q_m}} = H_m \xvec{k}_m.
\end{equation}

\begin{definition}
  Величины $H_m$ (иногда удобнее обозначение $H_{q_m}$) называют
  \textit{коэффициентами Ламе}.
\end{definition}

Пользуясь формулой \ref{eq:local_basis_vec}, легко найти косинусы углов
криволинейной системы координат с осями декартовых координат. Например,
\begin{equation*}
  \cos(\widehat{\xvec{k}_i, [x_1]}) = \dotprod{\xvec{k}_i}{\xvec{e}_1} =
    \frac{1}{H_i} \dotprod{\diffp{\xvec{r}}{{q_i}}}{\xvec{e}_1} =
    \frac{1}{H_i} \diffp{x_1}{{q_i}}
\end{equation*}
и, вообще,
\begin{equation*}
  \cos(\widehat{\xvec{k}_i, [x_j]}) = \frac{1}{H_i} \diffp{x_j}{{q_i}}.
\end{equation*}

Составим таблицу направляющих косинусов:
\begin{equation*}
  \setlength{\extrarowheight}{-2pt}
  \renewcommand{\arraystretch}{4}
  \begin{array}{|*4{>{\displaystyle}c|}}
    \hline
    & [q_1] & [q_2] & [q_3] \\
    \hline
    {[x_1]}
      & \frac{1}{H_1} \diffp{x_1}{{q_1}}
      & \frac{1}{H_2} \diffp{x_1}{{q_2}}
      & \frac{1}{H_3} \diffp{x_1}{{q_3}} \\
    {[x_2]}
      & \frac{1}{H_1} \diffp{x_2}{{q_1}}
      & \frac{1}{H_2} \diffp{x_2}{{q_2}}
      & \frac{1}{H_3} \diffp{x_2}{{q_3}} \\
    {[x_3]}
      & \frac{1}{H_1} \diffp{x_3}{{q_1}}
      & \frac{1}{H_2} \diffp{x_3}{{q_2}}
      & \frac{1}{H_3} \diffp{x_3}{{q_3}} \\
    \hline
  \end{array}
  \renewcommand{\arraystretch}{1}
\end{equation*}

\begin{definition}
  \textit{Движением точки в криволинейных координатах $\xvec{q}$} называют
  дважды непрерывно дифференцируемую на промежутке $J \subset \mathbb{R}$
  вектор-функцию $\xvec{q}(t)$.
\end{definition}

\begin{definition}
  Функции $\xvec[.]{q}$ и $\xvec[:]{q}$ называют соответственно
  \textit{обобщённой скоростью} и \textit{обобщённым ускорением точки в
  движении $\xvec{q}(t)$}.
\end{definition}

\begin{definition}
  Кривую
  \begin{equation*}
    \Gamma = \{ \xvec{q}(t) \in \mathbb{R}^3 \, | \, t \in J \}
  \end{equation*}
  называют \textit{траекторией точки в криволинейных координатах}.
\end{definition}

\subsection{Проекции скорости точки на оси криволинейной системы координат}

Напишем вектор скорости в виде
\begin{equation}
  \label{eq:velocity_def}
  \xvec{v} = \xvec[.]{r} = \diffp{\xvec{r}}{{q_1}} \dot{q_1} +
  \diffp{\xvec{r}}{{q_2}} \dot{q_2} + \diffp{\xvec{r}}{{q_3}} \dot{q_3},
\end{equation}
тогда по формулам \ref{eq:local_basis_vec_alt} получим
\begin{equation*}
  \xvec{v} = H_1 \dot{q}_1 \xvec{k}_1 + H_2 \dot{q}_2 \xvec{k}_2 + H_3
  \dot{q}_3 \xvec{k}_3.
\end{equation*}
Это равенство можно рассматривать как разложение вектора скорости по единичным
векторам осей криволинейных координат; для проекций скорости на координатные
оси будем иметь
\begin{equation}
  \label{eq:velocity_proj}
  v_{q_m} = H_{q_m} \dot{q}_m \quad (m = 1,2,3).
\end{equation}

Если криволинейная система ортогональна, то
\begin{equation}
  \begin{gathered}
    v = \sqrt{\paren{H_1 \dot{q_1}}^2 + \paren{H_2 \dot{q_2}}^2 + \paren{H_3
      \dot{q_3}}^2}, \\
    \cos\angle (\xvec{v}, \xvec{k}_m) = H_m \dot{q}_m v^{-1},
      \quad m = 1,2,3.
  \end{gathered}
\end{equation}

\subsection{Список литературы}
\begin{enumerate}
  \item \cite{lectures}
  \item \cite{lourie}
\end{enumerate}

