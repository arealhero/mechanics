\section{Движение точки по прямой и по окружности}

\subsection{Прямолинейное движение}

\begin{definition}
  \textit{Прямолинейное движение} --- движение точки, траектория которой лежит
  на прямой.
\end{definition}

Начало системы $Oxyz$ поместим на этой прямой, а ось $x$ направим вдоль неё;
получим уравнение траектории:
\begin{equation*}
  y = 0, \quad z = 0;
\end{equation*}
тогда
\begin{equation*}
  \begin{gathered}
    v^2 = (\dot{x}(t))^2, \\
    w^2 = (\ddot{x}(t))^2 \\
  \end{gathered}
\end{equation*}
и, как следствие,
\begin{equation*}
  \begin{gathered}
    w_\tau^2 = (\dot{v})^2 = (\ddot{x})^2, \quad
      w_n = \sqrt{w^2 - w_\tau^2} = 0, \\
    K = 0, \quad \rho = +\infty.
  \end{gathered}
\end{equation*}

\begin{definition}
  Прямолинейное движение называют \textit{равномерным}, если $v(t) = v_0$,
  где $v_0$ --- постоянная.

  Уравнение движения:
  \begin{equation*}
    x(t) = x_0 + v_0 (t - t_0), \quad x(t_0) = x_0.
  \end{equation*}

  Естественная координата:
  \begin{equation*}
    s = \abs{v_0 (t - t_0)}.
  \end{equation*}
\end{definition}

\begin{definition}
  Прямолинейное движение называют \textit{равнопеременным}, если $w(t) = w_0$,
  где $w_0$ --- постоянная.

  Уравнение движения:
  \begin{equation*}
    \begin{gathered}
      x(t) = x_0 + v_0 (t - t_0) + \frac{w_0}{2} (t - t_0)^2, \\
      x(t_0) = x_0, \quad \dot{x}(t_0) = v(t_0) = v_0.
    \end{gathered}
  \end{equation*}

  Естественная координата:
  \begin{equation*}
    s = \abs{v_0 (t - t_0) + \frac{w_0}{2} (t - t_0)^2}.
  \end{equation*}
\end{definition}

\subsection{Движение по окружности}

\begin{definition}
  \textit{Углом поворота между векторами} называется вектор
  \begin{equation*}
    \angle (\vec{a}, \vec{b}) = \left\{
      \begin{array}{l l}
        (\arccos(\vec{a}, \vec{b})) \cdot
        \frac{\crossprod{\vec{a}}{\vec{b}}}
        {\abs{\crossprod{\vec{a}}{\vec{b}}}}, &
        \vec{a} \nparallel \vec{b}; \\

        \vec{0}, & \vec{a} \parallel \vec{b}.
      \end{array}
      \right.
  \end{equation*}
\end{definition}

\begin{definition}
  \textit{Углом между векторами $\vec{a}$ и $\vec{b}$} называется величина
  \begin{equation*}
    \abs{\angle (\vec{a}, \vec{b})} = \arccos(\vec{a}, \vec{b}).
  \end{equation*}
\end{definition}

Когда говорят об угле между векторами $\vec{a}$ и $\vec{b}$, отсчитываемом от
$\vec{a}$ к $\vec{b}$, то имеют в виду угол поворота $\angle(\vec{a}, \vec{b})$.

\begin{definition}
  \textit{Движением по окружности} называют любое движение точки, траектория
  которого лежит на окружности.
\end{definition}

В случае движения по окружности угол смежности $\varepsilon$ равен центральному
углу между радиусами, проведёнными в точки касания, а соответствующая дуга
равна произведению этого угла на радиус $R$, то есть
\begin{equation*}
  \Delta \sigma = \varepsilon R, \implies
    \frac{\varepsilon}{\Delta \sigma} = \frac{1}{R},
\end{equation*}
поэтому
\begin{equation*}
  K = \lim_{\Delta \sigma \to 0} \frac{\varepsilon}{\Delta \sigma} =
    \frac{1}{R}, \quad \rho = R.
\end{equation*}

% TODO
(\textcolor{red}{TODO:} решить, куда поместить определения угловой скорости и
ускорения, а также скалярные и векторные формулы скорости и ускорнения точек)

\begin{definition}
  Движение по окружности называют \textit{равномерным вращением}, если
  $\omega(t) = \omega_0$, где $\omega_0$ --- постоянная.

  В этом случае
  \begin{equation*}
    \varphi(t) = \varphi_0 + \omega_0 (t - t_0), \quad \varphi(t_0) = \varphi_0.
  \end{equation*}
\end{definition}

\begin{definition}
  Движение по окружности называют \textit{равнопеременным вращением}, если
  $\varepsilon = \varepsilon_0$, где $\varepsilon_0$ --- постоянная.

  В этом случае
  \begin{equation*}
    \begin{gathered}
      \varphi(t) = \varphi_0 + \omega_0 (t - t_0) +
        \frac{\varepsilon_0}{2} (t - t_0)^2, \\
      \varphi(t_0) = \varphi_0, \quad
        \dot{\varphi(t_0)} = \omega(t_0) = \omega_0.
    \end{gathered}
  \end{equation*}
\end{definition}

Рассмотрим частные случаи движения по окружности:
\begin{enumerate}
  \item Если тело вращается равномерно, то $\varepsilon(t) = 0$, поэтому
    \begin{equation*}
      w_\tau = 0, \quad w_n = R \omega_0^2.
    \end{equation*}

  \item Если в некоторый момент времени угловая скорость $\omega$ тела достигает
    максимального или минимального значения, то
    $\dot{\omega} = \varepsilon = 0$, поэтому
    \begin{equation*}
      w_\tau = 0, \quad w_n = R \omega_0^2.
    \end{equation*}

  \item Если в некоторый момент угол поворота достигает максимального или
    минимального значения, то $\dot{\varphi} = \omega = 0$, поэтому
    \begin{equation*}
      w_\tau = 0, \quad w_n = 0.
    \end{equation*}
\end{enumerate}

\subsection{Список литературы}
\begin{enumerate}
  \item \cite{lectures}
  \item \cite{lourie}
\end{enumerate}

