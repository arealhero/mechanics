\section{Скорость точек твёрдого тела в общем случае}

Пусть $(O', \pvec{i}, \pvec{j}, \pvec{k})$ --- репер, жёстко связанный с твёрдым
телом (подвижный репер), а $(O, \vec{i}, \vec{j}, \vec{k})$ --- неподвижный
репер. Тогда если $(x_0, y_0, z_0)$ --- координаты точки $O'$ в неподвижном
репере, то связь между координатами произвольной точки $M$ тела в неподвижном и
подвижном реперах следующая:
\begin{equation}
  \left(
  \begin{array}{c}
    x \\
    y \\
    z
  \end{array}
  \right)
  =
  \left(
  \begin{array}{c}
    x_0 \\
    y_0 \\
    z_0
  \end{array}
  \right)
  +
  \left(
  \begin{array}{c c c}
    p_{11} & p_{21} & p_{31} \\
    p_{12} & p_{22} & p_{32} \\
    p_{13} & p_{23} & p_{33}
  \end{array}
  \right)
  \left(
  \begin{array}{c}
    x' \\
    y' \\
    z'
  \end{array}
  \right).
\end{equation}

Как видим, перемещение $\Delta \vec{r}$ точки $M$ складывается из перемещения
$\Delta \vec{r}_{O'}$ точки $O'$ и вращательного перемещения $\Delta(\vec{r} -
\vec{r}_{O'})$ точки $M$ при повороте тела вокруг $O'$, то есть
\begin{equation}
  \label{eq:general_point_movement_delta}
  \Delta \vec{r} = \Delta \vec{r}_{O'} + \Delta(\vec{r} - \vec{r}_{O'}),
\end{equation}
где
\begin{equation*}
  \Delta \vec{r}_{O'} = \vec{v}_{O'} \Delta t + \vec{o}(\Delta t)
    \mbox{~при~} \Delta t \to 0
\end{equation*}
и
\begin{equation*}
  \Delta(\vec{r} - \vec{r}_{O'}) = 
    \crossprod
      {\vv{\Delta \varphi_{O'}}}
      {(\vec{r} - \vec{r}_{O'})}
    \mbox{~при~} \Delta t \to 0.
\end{equation*}

Разделив равенство \ref{eq:general_point_movement_delta} на $\Delta t$ и перейдя
при $\Delta t \to 0$ к пределу, получим:
\begin{equation}
  \label{eq:general_point_velocity_temp}
  \vec{v} = \vec{v}_{O'} +
    \crossprod
      {\vec{\omega}_{O'}}
      {(\vec{r} - \vec{r}_{O'})}.
\end{equation}
Здесь
\begin{equation*}
  \vec{\omega}_{O'} = \lim_{\Delta t \to 0}
    \paren{
      \frac{\vv{\Delta \varphi_{O'}}}{\Delta t}
    }
\end{equation*}
означает мгновенную угловую скорость вращения тела вокруг точки $O'$, а
$\vec{v}$ и $\vec{v}_{O'}$ --- скорости точек $M$ и $O'$.

\begin{theorem}
  \label{theorem:general_velocity_pole_independence}
  Вектор $\vec{\omega}_{O'}$ не зависит от выбора полюса $O'$.
\end{theorem}

\begin{proof}
  Пусть $B$ --- другой полюс, тогда
  \begin{equation*}
    \begin{aligned}
      \vec{v} &= \vec{v}_B +
        \crossprod{\vec{\omega}_B}{(\vec{r} - \vec{r}_B)}, \\
      \vec{v}_B &= \vec{v}_{O'} +
        \crossprod{\vec{\omega}_{O'}}{(\vec{r}_B - \vec{r}_{O'})},
    \end{aligned}
  \end{equation*}
  откуда получаем
  \begin{equation}
    \label{eq:double_pole_velocity}
    \vec{v} = \vec{v}_{O'}
        + \crossprod{\vec{\omega}_{O'}}{(\vec{r}_B - \vec{r}_{O'})}
        + \crossprod{\vec{\omega}_B}{(\vec{r} - \vec{r}_B)}.
  \end{equation}

  Вычитая из равенства \ref{eq:general_point_velocity_temp} равенство
  \ref{eq:double_pole_velocity}, получаем
  \begin{equation*}
    \crossprod{\vec{\omega}_{O'}}{(\vec{r} - \vec{r}_{O'})}
      - \crossprod{\vec{\omega}_{O'}}{(\vec{r}_B - \vec{r}_{O'})}
      - \crossprod{\vec{\omega}_B}{(\vec{r} - \vec{r}_B)} = \vec{0},
  \end{equation*}
  то есть
  \begin{equation*}
    \crossprod
      {(\vec{\omega}_{O'} - \vec{\omega}_B)}
      {(\vec{r} - \vec{r}_B)} = \vec{0}.
  \end{equation*}

  Так как это равенство истинно для любого $\vec{r}$, то $\vec{\omega}_{O'} =
  \vec{\omega}_B$.
\end{proof}

Согласно теореме \ref{theorem:general_velocity_pole_independence} вектор
$\vec{\omega}_{O'}$ можно обозначить просто $\vec{\omega}$ --- это
\textit{угловая скорость твёрдого тела} в общем случае. Формула
\ref{eq:general_point_velocity_temp} запишется в следующем виде:
\begin{equation}
  \label{eq:general_euler_formula}
  \vec{v} = \vec{v}_{O'} +
    \crossprod
      {\vec{\omega}}
      {(\vec{r} - \vec{r}_{O'})}.
\end{equation}
Эту формулу называют \textit{формулой Эйлера в общем случае}.

\begin{corollary}
  Проекции скоростей любых двух различных точек абсолютно твёрдого тела на
  направление соединяющего их отрезка равны между собой.
\end{corollary}

\subsection{Список литературы}
\begin{enumerate}
  \item \cite{lectures}
\end{enumerate}

