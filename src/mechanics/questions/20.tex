\section{Две геометрические теоремы о движении твёрдого тела вокруг неподвижной
точки}

\begin{definition}
  Перемещение твёрдого тела такое, что начальное и конечное положения кажддой
  его точки совпадают, называют \textit{нулевым}.
\end{definition}

\begin{theorem}[Эйлера-Даламбера]
  \label{theorem:euler_d'alembert}
  Для любого ненулевого перемещения $\Pi$ твёрдого тела вокруг неподвижной точки
  существует единственная прямая $l$ такая, что перемещение $\Pi$ можно
  представить как перемещение в результате поворота этого тела вокруг этой оси
  на некоторый угол $\alpha$.
\end{theorem}

\begin{proof}
  Пусть $O$ --- неподвижная точка тела. Будем считать, что в начальном положении
  подвижный и неподвижный реперы $(O, \vec{i}, \vec{j}, \vec{k})$ и $(O,
  \vec{e}_{x'}, \vec{e}_{y'}, \vec{e}_{z'})$ совпадают. Пусть $(x_1', y_1',
  z_1')$ и $(x_2', y_2', z_2')$ --- координаты в неподвижном репере произвольной
  точки $M$ твёрдого тела (пространства, связанного с этим телом) в его
  начальном и конечном положениях соответственно, а $(x,y,z)$ --- координаты
  этой точки в подвижном репере. Используя формулы \ref{eq:coords_transform},
  получаем
  \begin{equation*}
    \begin{aligned}
      \left(
      \begin{array}{c}
        x_1' \\
        y_1' \\
        z_1'
      \end{array}
      \right)
      &= I \cdot \left(
      \begin{array}{c}
        x \\
        y \\
        z \\
      \end{array}
      \right), \\
      \left(
      \begin{array}{c}
        x_2' \\
        y_2' \\
        z_2'
      \end{array}
      \right)
      &= P \cdot \left(
      \begin{array}{c}
        x \\
        y \\
        z \\
      \end{array}
      \right),
    \end{aligned}
  \end{equation*}
  где $I$ --- единичная матрица, $P$ --- ортогональная, $\det P = 1$ и
  $P \neq I$.

  Необходимо показать, что множество точек $M \sim (x,y,z)$, удовлетворяющих
  равенству $(x_1', y_1', z_1') = (x_2', y_2', z_2')$, то есть равенству
  \begin{equation*}
    (P - I) \cdot \left(
    \begin{array}{c}
      x \\
      y \\
      z
    \end{array}
    \right) = \left(
    \begin{array}{c}
      0 \\
      0 \\
      0
    \end{array}
    \right),
  \end{equation*}
  суть множество всех точек некоторой прямой, проходящей через $(0,0,0)$.

  Эту задачу можно переформулировать так: мы должны доказать, что среди
  собственных значений $\lambda$ матрицы $P$ есть значение $\lambda_1 = 1$, и
  ему соответствует одномерное подпространство собственных векторов. Чтобы
  сделать это, мы покажем, что $\lambda_1 = 1$ является корнем
  характеристического полинома $d(\lambda) = \det \paren{P - \lambda I}$ и что
  кратность этого корня равна единице.

  Действительно, из цепочки равенств
  \begin{equation*}
    \begin{aligned}
      d(1) &= \det(P - I) \\
      &= \det(P^T - I^T) \\
      &= \det(P^{-1} - I) \\
      &= \det(P \cdot (P^{-1} - I)) \\
      &= \det(I - P) \\
      &= \det(-(P - I)) \\
      &= (-1)^3 \det(P - I) \\
      &= -d(1)
    \end{aligned}
  \end{equation*}
  следует, что величина $\lambda_1 = 1$ является корнем полинома $d(\lambda)$.

  % TODO
  (\textcolor{red}{TODO:} доказать, что кратность $\lambda_1 = 1$ равна единице)
\end{proof}

\begin{definition}
  Прямую $l$ называют \textit{осью вращения}.
\end{definition}

Пусть $M$ --- произвольная точка твёрдого тела, движущегося вокруг неподвижной
точки $O$, а $\vec{r}(t)$ и $\vec{r}(t + \Delta t)$ --- вектор-радиусы этой
точки в моменты времени $t$ и $t + \Delta t$. Вектор
\begin{equation*}
  \Delta \vec{r} = \vec{r}(t + \Delta t) - \vec{r}(t)
\end{equation*}
соответствует перемещению точки $M$ за время от $t$ до $t + \Delta t$. В
соответствии с теоремой \ref{theorem:euler_d'alembert}, его можно вычислить как
перемещение при вращении вокруг некоторой оси на угол $\vv{\Delta \varphi}$
(этот вектор направлен вдоль упомянутой оси согласно определению угла поворота).
Определяя скорость $\vec{v}$ как предел при $\Delta t \to 0$ отношения малого
перемещения $\Delta \vec{r}$ к промежутку времени $\Delta t$, найдём
\begin{equation*}
  \vec{v} = \lim_{\Delta t \to 0} \frac{\Delta \vec{r}}{\Delta t} = \lim_{\Delta
    t \to 0} \paren{\crossprod{\frac{\vv{\Delta \varphi}}{\Delta t}}{\vec{r}}}
    = \crossprod{\paren{\lim_{\Delta t \to 0} \frac{\vv{\Delta \varphi}}{\Delta
    t}}}{\vec{r}}.
\end{equation*}

Вводя \textit{вектор мгновенной угловой скорости}
\begin{equation*}
  \vec{\omega} = \lim_{\Delta t \to 0} \frac{\vv{\Delta \varphi}}{\Delta t},
\end{equation*}
получим
\begin{equation*}
  \vec{v} = \crossprod{\vec{\omega}}{\vec{r}}.
\end{equation*}

\begin{definition}
  Прямую, проходящую через точку $O$ и параллельную вектору мгновенной угловой
  скорости $\vec{\omega}$, называют \textit{мгновенной осью вращения твёрдого
  тела в момент времени $t$}.
\end{definition}

\begin{theorem}
  Угловую скорость твёрдого тела с неподвижной точкой можно вычислить по формуле
  \begin{equation}
    \label{eq:immovable_point_angular_velocity}
    \vec{\omega} = \dot{\psi} \vec{k} + \dot{\theta} \vec{n} + \dot{\varphi}
      \pvec{k}.
  \end{equation}
\end{theorem}

\begin{proof}
  % TODO
  (\textcolor{red}{TODO:} доказать (в книге есть))
\end{proof}

\begin{definition}
  Геометрическое место точек мгновенных осей вращения в неподвижном и подвижном
  реперах называют соответственно \textit{неподвижным} и \textit{подвижным
  аксоидом}.
\end{definition}

\begin{theorem}[Пуансо]
  При движении твёрдого тела, имеющего неподвижную точку, подвижный аксоид
  катится без скольжения по неподвижному.
\end{theorem}

\begin{proof}
  % TODO
  (\textcolor{red}{TODO:} доказать (в книге есть))
\end{proof}

\subsection{Список литературы}
\begin{enumerate}
  \item \cite{lectures}
  \item \cite{lourie}
\end{enumerate}

