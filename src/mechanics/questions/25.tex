\section{Сложное движение точки, основные понятия}

Пусть $(O, \xvec{i}, \xvec{j}, \xvec{k})$ и $(O', \xvec{i}', \xvec{j}',
\xvec{k}')$
--- неподвижный и подвижный реперы. Эти реперы и связанные с ними подвижное и
неподвижное пространства называют также \textit{абсолютным} и
\textit{относительным} соответственно.

\begin{definition}
  Движение, скорость и ускорение точки $M$ относительно неподвижного
  (абсолютного) репера называют \textit{абсолютным}.
\end{definition}

\begin{definition}
  Движение, скорость и ускорение точки $M$ относительно подвижного
  (относительного) репера называют \textit{относительным}.
\end{definition}

\begin{definition}
  В момент $t$ точка $M$ совпадает с точкой $M'$ подвижного пространства.
  Движение, скорость и ускорение точки $M'$ в момент времени $t$ относительно
  абсолютного репера называют \textit{переносными} для точки $M$ в этот момент.
\end{definition}

Будем использовать следующие обозначения:
\begin{itemize}
  \item $\xvec{r} = \xvec{OM}$ --- абсолютный вектор-радиус;
  \item $\xvec{v}$ --- абсолютная скорость;
  \item $\xvec{w}$ --- абсолютное ускорение;
  \item $\xvec{\rho} = \xvec{O'M}$ --- относительный вектор-радиус;
  \item $\xvec{v}_r$ --- относительная скорость;
  \item $\xvec{w}_r$ --- относительное ускорение;
  \item $\xvec{v}_e$ --- переносная скорость;
  \item $\xvec{w}_e$ --- переносное ускорение;
  \item $\xvec{\omega}$ --- угловая скорость подвижного репера относительно
    неподвижного
\end{itemize}

Пусть $\xvec{C}$ --- вектор-функция аргумента $t$, причём
\begin{equation}
  \xvec{C} = C_x \xvec{i} + C_y \xvec{j} + C_z \xvec{k}.
\end{equation}
Тогда
\begin{equation}
  \label{eq:vector_function_full_derivative}
  \xvec[.]{C} = \dot{C_x} \xvec{i} + \dot{C_y} \xvec{j} + \dot{C_z} \xvec{k}
    + C_x \xvec[.]{i} + C_y \xvec[.]{j} + C_z \xvec[.]{k}.
\end{equation}

Производные $\xvec[.]{i},~\xvec[.]{j},~\xvec[.]{k}$ зависят от
пространства, в котором они рассматриваются. В частности, в подвижном
пространстве они равны нулю.

\begin{theorem}[Формулы Пуассона]
  Пусть подвижный репер $(O', \xvec{i}', \xvec{j}', \xvec{k}')$, жёстко
  связанный с твёрдым телом, движется относительно неподвижного репера
  $(O, \xvec{i}, \xvec{j}, \xvec{k})$ с угловой скоростью $\xvec{\omega}$. Тогда
  производные подвижных ортов в неподвижном репере вычисляются по формулам:
  \begin{equation}
    \label{eq:poisson_formulas}
    \begin{aligned}
      \xvec[.]{i}' &= \crossprod{\xvec{\omega}}{\xvec{i}'}, \\
      \xvec[.]{j}' &= \crossprod{\xvec{\omega}}{\xvec{j}'}, \\
      \xvec[.]{k}' &= \crossprod{\xvec{\omega}}{\xvec{k}'}.
    \end{aligned}
  \end{equation}
\end{theorem}

\begin{proof}
  Докажем первую формулу; остальные доказываются аналогично. Введём обозначения
  \begin{equation*}
    \xvec{r}_{O'} = \xvec{OO'}, \quad
    \xvec{v}_{O'} = \xvec[.]{r}_{O'}, \quad
    \xvec{r}_{O_{i'}} = \xvec{r}_{O'} + \xvec{i}, \quad
    \xvec{v}_{O_{i'}} = \xvec[.]{r}_{O_{i'}}.
  \end{equation*}

  Дифференцированием равенства $\xvec{r}_{O_{i'}} = \xvec{r}_{O'} + \xvec{i}'$
  получаем
  \begin{equation*}
    \xvec{v}_{O_{i'}} = \xvec{v}_{O'} + \xvec[.]{i}'.
  \end{equation*}

  По формуле Эйлера имеем
  \begin{equation*}
    \xvec{v}_{O_{i'}} = \xvec{v}_{O'} +
      \crossprod{\xvec{\omega}}{(\xvec{r}_{O_{i'}} - \xvec{r}_{O'})}
    = \xvec{v}_{O'} + \crossprod{\xvec{\omega}}{\xvec{i}'},
  \end{equation*}
  следовательно,
  \begin{equation*}
    \xvec[.]{i}' = \crossprod{\xvec{\omega}}{\xvec{i}'},
  \end{equation*}
  что и требовалось доказать.
\end{proof}

\begin{definition}
  Производную вектор-функции $\xvec{C}$ в подвижном репере $(O', \xvec{i}',
  \xvec{j}', \xvec{k}')$ называют \textit{относительной производной
  вектор-функции $\xvec{C}$} и обозначают $\frac{d' \xvec{C}}{dt}$.
\end{definition}

\begin{definition}
  Производную вектор-функции $\xvec{C}$ в неподвижном репере $(O, \xvec{i},
  \xvec{j}, \xvec{k})$ называют \textit{абсолютной производной вектор-функции
  $\xvec{C}$} и обозначают $\dt[\xvec{C}]$.
\end{definition}

\begin{theorem}[Формула относительной производной]
  \label{theorem:relative_derivative}
  Относительная и абсолютная производные вектор-функции связаны равенством:
  \begin{equation}
  \label{eq:relative_derivative}
    \dt[\xvec{C}] = \frac{d' \xvec{C}}{dt} +
      \crossprod{\xvec{\omega}}{\xvec{C}}.
  \end{equation}
\end{theorem}

\begin{proof}
  Формулу \ref{eq:vector_function_full_derivative} перепишем в новых
  обозначениях:
  \begin{equation*}
    \dt[\xvec{C}] = \frac{d' \xvec{C}}{dt}
      + C_x \dt[\xvec{i}] + C_y \dt[\xvec{j}] + C_z \dt[\xvec{k}].
  \end{equation*}
  
  Используя формулы Пуассона \ref{eq:poisson_formulas}, получим
  \begin{equation*}
    \begin{aligned}
      \dt[\xvec{C}] &= \frac{d' \xvec{C}}{dt}
        + \crossprod
          {\xvec{\omega}}
          {\paren{C_x \xvec{i} + C_y \xvec{j} + C_z \xvec{k}}} \\
      &= \frac{d' \xvec{C}}{dt} + \crossprod{\xvec{\omega}}{\xvec{C}}.
    \end{aligned}
  \end{equation*}
\end{proof}

\subsection{Список литературы}
\begin{enumerate}
  \item \cite{lectures}
\end{enumerate}

