\section{Проекции ускорения точки на оси ортогональной криволинейной системы
координат}

Для определения проекций ускорения представим их в виде
\begin{equation*}
  w_{q_m} = \dotprod{\vec{w}}{\vec{\tau}_m} =
    \dotprod{\dot{\vec{v}}}{\frac{1}{H_m}}{\parder[\vec{r}]{q_m}},
\end{equation*}
откуда
\begin{equation}
  \label{eq:accel_proj_temp}
  H_m w_{q_m} = \dotprod{\dot{\vec{v}}}{\parder[\vec{r}]{q_m}} =
    \dt \paren{\dotprod{\vec{v}}{\parder[\vec{r}]{q_m}}}
    - \dotprod{\vec{v}}{\dt \parder[\vec{r}]{q_m}}.
\end{equation}
Из \autoref{eq:velocity_def} непосредственно следует
\begin{equation}
  \label{eq:accel_dr}
  \parder[\vec{v}]{\dot{q}_m} = \parder[\vec{r}]{q_m}.
\end{equation}
Кроме того, по определению полной производной
\begin{equation*}
  \dt \parder[\vec{r}]{q_m} = \pparder[\vec{r}]{q_1}{q_m} \dot{q}_1 +
    \pparder[\vec{r}]{q_2}{q_m} \dot{q}_2 +
    \pparder[\vec{r}]{q_3}{q_m} \dot{q}_3;
\end{equation*}
но это же выражение получим, если возьмём от обеих частей
\autoref{eq:velocity_def} частную производную по $q_m$. Действительно, так как
$\dot{q}_1,~\dot{q}_2,~\dot{q}_3$ зависят только от времени, а не от
$q_1,~q_2,~q_3$, то
\begin{equation*}
  \parder[\vec{v}]{q_m} = \pparder[\vec{r}]{q_m}{q_1} \dot{q}_1 +
    \pparder[\vec{r}]{q_m}{q_2} \dot{q}_2 + \pparder[\vec{r}]{q_m}{q_3}
    \dot{q}_3;
\end{equation*}
таким образом, имеем
\begin{equation}
  \label{eq:accel_dv}
  \dt \parder[\vec{r}]{q_m} = \parder[\vec{v}]{q_m}.
\end{equation}
Подставляя значения $\parder[\vec{r}]{q_m}$ по \autoref{eq:accel_dr} и
$\dt \parder[\vec{r}]{q_m}$ по \autoref{eq:accel_dv} в равенство
\ref{eq:accel_proj_temp}, получим
\begin{equation}
  \label{eq:accel_proj_temp2}
  H_m w_{q_m} = \dt \paren{\dotprod{\vec{v}}{\parder[\vec{v}]{\dot{q}_m}}} -
    \dotprod{\vec{v}}{\parder[\vec{v}]{q_m}}.
\end{equation}
Замечая, что
\begin{equation*}
  \begin{gathered}
    \dotprod{\vec{v}}{\parder[\vec{v}]{\dot{q}_m}} =
      \parder{\dot{q}_m} \frac{\dotprod{\vec{v}}{\vec{v}}}{2} =
      \parder{\dot{q}_m} \frac{v^2}{2}, \\
    \dotprod{\vec{v}}{\parder[\vec{v}]{q_m}} =
      \parder{q_m} \frac{\dotprod{\vec{v}}{\vec{v}}}{2} =
      \parder{q_m} \frac{v^2}{2},
  \end{gathered}
\end{equation*}
на основании \autoref{eq:accel_proj_temp2} получим выражение проекций ускорения
на оси криволинейной системы координат:
\begin{equation}
  w_{q_m} = \frac{1}{H_m} \paren{\dt \parder[T]{\dot{q}_m} - \parder[T]{q_m}},
\end{equation}
где для краткости введено обозначение
\begin{equation}
  T = \frac{1}{2} v^2.
\end{equation}
Используя линейный дифференциальный оператор Эйлера-Лагранжа, определяемый
формулой
\begin{equation}
  E_{q_m}(T) = \dt \parder[T]{\dot{q}_m} - \parder[T]{q_m},
\end{equation}
окончательно получаем
\begin{equation}
  \label{eq:accel_proj}
  w_{q_m} = \frac{1}{H_{q_m}} E_{q_m}(T).
\end{equation}

\subsection{Список литературы}
\begin{enumerate}
  \item \cite{lourie}
\end{enumerate}

